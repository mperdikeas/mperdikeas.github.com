%%%%%%%%%%%%%%%%%%%%%%%%%%%%%%%%%%%%%%%%%
% Cies Resume/CV
% LaTeX Template
% Version 1.1 (20/7/14)
%
% This template has been downloaded from:
% http://www.LaTeXTemplates.com
%
% Original author:
% Cies Breijs (cies@kde.nl)
% https://github.com/cies/resume with extensive modifications by:
% Vel (vel@latextemplates.com)
%
% License:
% CC BY-NC-SA 3.0 (http://creativecommons.org/licenses/by-nc-sa/3.0/)
%
%%%%%%%%%%%%%%%%%%%%%%%%%%%%%%%%%%%%%%%%%

%----------------------------------------------------------------------------------------
%	PACKAGES AND OTHER DOCUMENT CONFIGURATIONS
%----------------------------------------------------------------------------------------

\documentclass[10pt,a4paper]{article} % Font size (10-12pt) and paper size (a4paper, letterpaper, legalpaper, etc)

\include{structure} % Include structure.tex which contains packages and document layout definitions

\hyphenation{Some-long-word} % Specify custom hyphenation points in words with dashes where you would like hyphenation to occur, or alternatively, don't put any dashes in a word to stop hyphenation altogether

\newcommand{\technologies}[0]{\textbf{\textit{Technologies:}}}
\newcommand{\ts}{\textsuperscript}
\newcommand\hairspace{\kern .08333em }
\usepackage[super]{nth}
\begin{document} 

%----------------------------------------------------------------------------------------
%	NAME AND CONTACT INFORMATION
%----------------------------------------------------------------------------------------

\maintitle{Yannis Koufoudakis}{%December 24, 1974
}  % Your name and date of birth or subtitle

\noindent\href{mailto:koufoudakisy@gmail.com}{koufoudakisy@gmail.com}\bull % Your email address
%\textsmaller{+}30 (210) 5322914\bull
koufoudakis \textit{(Skype)}\bull % Your phone number(s) and Skype username
% \href{http://mperdikeas.github.com}{mperdikeas.github.com}\\ % Your URL
Nestanis 13, Peristeri, 12135, Greece % Your address

\spacedhrule{0.9em}{-0.4em} % Horizontal rule - the first bracket is whitespace before and the second is after

%----------------------------------------------------------------------------------------
%	SUMMARY SECTION
%----------------------------------------------------------------------------------------

\roottitle{Summary} % Root section title

\vspace{-1.3em} % Reduce whitespace after the Summary heading and the two-column content

\begin{multicols}{2}  % Start a two-column layout
  \noindent
  %\textit{A summary of your interests, achievements, history, topic of study or any other short summary of your professional life.}\\\\
  Results-driven chief executive officer with over 12 years experience, leading and increasing growth
  in small and medium businesses. Proven ability to overcome complex business challenges and make high
  stakes decisions. Effective and accountable in high-profile and high-stress executive roles. Delivering
  value through relentless pursuit of growth, profitability, and customer satisfaction.
  Respect and leverage human capital, motivate, mentor and lead talented professionals.

  \begin{itemize}
  \item    Visionary, Strategy, Execution & Leadership
\item Business development
\item P/L & Performance improvement,
\item Pricing/Contract Negotiation
\item Define, design, manage and communicate requirements in the process of product development.
\item Lead by example
\item Motivational leadership that spurs people to willingly give their best and inspires loyalty.
\item Financial and Software background
\item Keynote Speaker in conferences (Greece and abroad)
    \end{itemize}


\end{multicols}

\spacedhrule{0.5em}{-0.4em} % Horizontal rule - the first bracket is whitespace before and the second is after

%----------------------------------------------------------------------------------------
%	EXPERIENCE SECTION
%----------------------------------------------------------------------------------------

\roottitle{Experience \& Achievements} % Top level section

\headedsection % Employer name which can include a hyperlink and location/URL on the right side of the page
{\href{https://www.neuropublic.gr}{Neuropublic Software and Telecommunications S.A.}} {
%    {\href{https://www.cfa.harvard.edu/}{Harvard-Smithsonian Center for Astrophysics}} {


\headedsubsection % Job title entry for the current employer
{Chief Executive Officer}
{2018 -- January 2019}
{\bodytext{Neuropublic is an innovative Informatics and Technology company founded in 2003 in Piraeus,
    employing about 100 people, specialising in the development of Web, Cloud-based integrated
    information systems and applications. Neuropublic occupies a leading position in the Agricultural
    sector and provides a wide range of technological services for the private and public sector. We
    developed the first and only, to date, large-scale Internet of Things Infrastructure deployment
    in Greece, installing thousands of wireless sensors on private agrigulultural land.
}}
}



\headedsection
    {\href{http://www.esac.com}{GAIA EPICHEIREIN Digital Services S.A}} {
\headedsubsection
{Chief Executive Officer}
{2014 -- 2018}
{\bodytext{
    GAIA EPICHEIREIN is a pioneering and innovative digital services provider company
    established in February 2014 as a result of a broad coalition between major players active
    in the agricultural business in Greece: 71 Agricultural Cooperatives all over Greece;
    Neuropublic S.A., and the Piraeus Bank Group. The turnover of GAIA EPICHEIREIN for 2018
    was in the order of 30 million Euros. Under my leadership, the Company diversified its
    revenue streams, increased its client base, expanded the original business plan, and
    established a nationwide network of 108 Farmers Service Centres (FSC). In 2016, GAIA
    EPICHEIREIN was awarded the EFQM Excellence Award and in 2017 it received the
    distinction ``EFQM -- Committed to Excellence 2''.
}}
}

\headedsection
    {\href{https://www.neuropublic.gr}{Neuropublic Software and Telecommunications S.A.}}
{

\headedsubsection
{Deputy Chief Executive Officer}
{2013 -- 2014}
{\bodytext{
    Neuropublic developed a cloud-based platform that offers a suite of applications for the agricultural sector.
    These include the electronic farmer card, ERP, CRM, MIS for cooperatives,  traceability application for
    agriculture products, smart irrigation and fertilization solutions etc. The central data repository
    is an information discovery and retrieval service that covers all agricultural activities and stakeholders
    in Greece. Information is gathered (mostly) via fully automated pipelines by harvesting data from web
    sites, publicly available web services, academic publications and solicited articles written by various
    contributors. Users can access the repository through a web portal or via web services to allow for
    unmediated software integration.

}}





\headedsubsection
{Head of International IT Projects}
{2012 -- 2013}
{\bodytext{
    Neuropublic developed a cloud-based platform that offers a suite of applications for the agricultural sector.
    These include the electronic farmer card, ERP, CRM, MIS for cooperatives,  traceability application for
    agriculture products, smart irrigation and fertilization solutions etc. The central data repository
    is an information discovery and retrieval service that covers all agricultural activities and stakeholders
    in Greece. Information is gathered (mostly) via fully automated pipelines by harvesting data from web
    sites, publicly available web services, academic publications and solicited articles written by various
    contributors. Users can access the repository through a web portal or via web services to allow for
    unmediated software integration.

}}

}



\headedsection
    {\href{https://www.neuropublic.gr}{Semantix information Technologies S.A.}}
{

\headedsubsection
{Chief Executive Officer}
{2008 -- 2012}
{\bodytext{
    Chief Executive Officer  (CEO) and co-founder of Semantix information Technologies S.A. Semantix
    was an boutique software development company founded in 2001. The company included, among
    its clients, the following organizations: European Space Agency (ESA), Vodafone (Greece), EADS
    Astrium SAS, Mobily (Saudi Arabia), Amdocs (USA), Oracle (USA), SwitchLab (UK), O2 (Ireland),
    Telefonica (Equador), Xerox (Greece), Byte (Greece), Fts-soft (Israel) and Bell (Canada). I
    led the company through the 2008 financial crisis and further developed its business model to
    include as clients a number of Greek public sector entities. I also initiated and handled the
    acquisition discussions that conluded in 2012 when Semantix was merged with Neuropublic SA.
}}



\headedsubsection
{Chief Executive Officer}
{2005 -- 2008}
{\bodytext{
Member of the board and IT Director in Semantix information Technologies S.A.
}}


\headedsubsection
{Senior Software Engineer}
{2003 -- 2005}
{\bodytext{
Software design and development and product management in Semantix information Technologies S.A. department of E-Government Projects.
}}




}








%------------------------------------------------

\headedsection
{\href{http://www.semantix.gr}{Semantix}}
{\textsc{Athens, Greece}} {

\headedsubsection % Job title entry for the current employer
{Technical Project Manager for the EGOS Visualization Tool}
{March '08 -- Sep '08}
{\bodytext{
I was technical project manager for the EGOS Visualization Tool for ESOC (Contract C21283). The project implemented a graphical front-end that integrates and provides additional functionality on top of the existing tools used by ESOC to check compliance with coding standards and conventions.
}}

\headedsubsection % Job title entry for the current employer
{Software Engineer in \href{http://www.tapeditor.com/}{Roaming Studio} products}
{2005 -- 2007}
{\bodytext{
    I was responsible for the implementation of the critical 
    conversions functionality in both the Roaming Studio and the Roaming Components products. These products
    are the company's flagship telecom products and revolve around the processing of TAP
    files which are used in \href{https://en.wikipedia.org/wiki/Roaming}{GSM Roaming}.
    The conversion logic is responsible for converting, e.g., a TAP file of version TAP3.11 into a file
    of version TAP3.10. The files are defined using
    different \href{https://en.wikipedia.org/wiki/Abstract_Syntax_Notation_One}{ASN.1} grammars both from
    a syntax and a semantics point of view so
    the transformation is a complex business logic procedure. Since there are more than 7 different \href{http://what-when-how.com/roaming-in-wireless-networks/transferred-account-procedures-billing-and-settlement/}{TAP3}
    versions and conversion had to take place between any arbitrary pair there was
    a large number of conversion routines to be written. I defined a
    \href{https://en.wikipedia.org/wiki/Domain-specific_language}{DSL} used to describe the logic
    behind these conversions (essentially transformation algorithms on deeply nested ASN.1 trees) and built a Python
    script that automatically generated C++ code implementing these transformations. In total I was able to
    reduce tens of thousands of lines of C++ code into as little as 700 conversion rules expressed in the
    above DSL
    (plus the Python script code generator).
    \\
    \technologies{} ASN.1, C++, Python
}}

\headedsubsection % Job title entry for the current employer
{Software Engineer at project ATLAS}
{2002 -- 2004}
{\bodytext{
    I was part of the engineering team that was tasked to implement the new Vodafone Greece billing system.
    Data (customer, subscriber, bills and
    calls information) migration to the new billing system and integration with over a dozen legacy
    peripheral systems which were scheduled to survive the migration to the new billing system.
    I was responsible for the integration code which was implemented in
    C++ and Java (for higher-level daemons) and which undertook to sustain the information flow between the
    new billing system and legacy systems which were not going to be replaced. Since no modifications were
    permitted in the legacy systems, the new billing system had to be wrapped in a fa\c{c}ade allowing it to
    expose the same interface towards the legacy systems.
    I was also involved with the design of a massive ``mediating'' staging database of more
    than 400 tables and 40,000 lines of \href{https://en.wikipedia.org/wiki/PL/SQL}{PL-SQL} code
    (a large percentage of which I implemented myself).
    Finally I had small exposure with \href{https://en.wikipedia.org/wiki/Portal_Software}{Portal Infranet} programming
    (now \href{http://www.orafaq.com/wiki/BRM}{Oracle Billing and Revenue Management}).
    \\
    \technologies{} Java, Oracle, PL-SQL, C++, C
}}

\headedsubsection % Job title entry for the current employer
{Software Engineer at project DCH}
{2001 -- 2002}
{\bodytext{
    Design and implementation of the Vodafone Greece Data Clearing House system for roaming calls.
    I implemented code in C++ (for the decoding / encoding and validations / transformations
    of the \href{http://what-when-how.com/roaming-in-wireless-networks/transferred-account-procedures-billing-and-settlement/}{TAP} records), Java (for the orchestration logic) and PL/SQL (for server-side processing)
    \\
    \technologies{} C++, ASN.1, Java, Oracle, PL-SQL.
}}


}
%------------------------------------------------


%------------------------------------------------

\begin{center}
%\textit{Please refer to \href{http://www.linkedin.com/in/ciesbreijs}{my Linkedin profile} for the complete list of work experiences along with recommendations.}
\end{center}

%------------------------------------------------

\spacedhrule{-0.2em}{-0.4em} % Horizontal rule - the first bracket is whitespace before and the second is after

%----------------------------------------------------------------------------------------
%	EDUCATION SECTION
%----------------------------------------------------------------------------------------

\roottitle{Education} % Top level section

\headedsection % Employer name which can include a hyperlink and location/URL on the right side of the page
{National Technical University of Athens}
{\textsc{Athens, Greece}} {

\headedsubsection % Job title entry for the current employer
{Ph.D. from the Department of Electrical and Computer Engineering}
{1998 -- 2001}
{\bodytext{
    Thesis Title: \textbf{`Distributed Processing Technologies for Unified Provision of Telecommunications and Internet Services'}.\\
    NTUA is the oldest and most prestigious technical university in Greece.
    I was awarded a Ph.D. for research into applying modern (at that time) software technologies
    in telecommunications networks. During that period of time I worked mostly with C++ and Java and developed
    code using distributed technologies like CORBA, RMI; Java applets, code interacting with telecom switch
    equipment, developed specialized video streaming applications in C++, etc.
    During that period of time I also published about
    a dozen papers (3 in international refereed journals including IEEE, the rest in conferences). Publications
    list provided at the end.
    }}
}

\headedsection % Employer name which can include a hyperlink and location/URL on the right side of the page
{Technical University of Patras}
{\textsc{Patras, Greece}} {

\headedsubsection % Job title entry for the current employer
{Engineering Diploma from the Computer Engineering and Informatics Department}
{1992 -- 1997}
{\bodytext{
    CEID is the oldest university department in Greece focusing exclusively in computer and software engineering
    and awarding an engineering diploma after 5 years of studies.
    I entered 1\ts{st} in rank after a nationwide competitive examination and finished 2\ts{nd} in rank
    (class size \textasciitilde{150}).
    Coded various applications in Pascal, Lisp, C and Java.
    }}
}


\spacedhrule{0.5em}{-0.4em} % Horizontal rule - the first bracket is whitespace before and the second is after

%----------------------------------------------------------------------------------------
%	SKILLS SECTION
%----------------------------------------------------------------------------------------

\roottitle{Skills} % Top level section

\inlineheadsection % Special section that has an inline header with a 'hanging' paragraph
{Technical specialties:}
{Software design and implementation, alone or in a team. I am a professional Java / JavaScript coder but 
  have also written production code in Python, Clojure and (in another life) C and C++.
  I am very comfortable
  at the command line and casually write Bash one-liners daily.\\
  \underline{Solid knowledge of the following technologies}: Java 7 (SE and EE), Ant/Ivy, HTML/CSS/JavaScript (ECMAScript 6 and 7), ReactJS, Spring Web MVC,
  \acr{XML}, \acr{XPath}, \acr{XSD}, \acr{REST}, \acr{SOAP}, relational databases (PostgreSQL, Oracle, MySQL, Sybase).
  \underline{Administration skills}:\ Linux administration, Bash scripting, Apache Tomcat, JBoss, PostgresSQL.
  \underline{Source control}:  Git, Subversion, ClearCase.
}

\inlineheadsection{}{}
%------------------------------------------------

\inlineheadsection % Special section that has an inline header with a 'hanging' paragraph
{Natural languages:}
{Greek \textit{(mother tongue)}, English \textit{(full professional proficiency: Cambridge Certificate of Proficiency in English, Grade A)}, Spanish \textit{(A1 level--no degree)}}

\spacedhrule{1.6em}{-0.4em} % Horizontal rule - the first bracket is whitespace before and the second is after

%----------------------------------------------------------------------------------------
%	REFERENCES SECTION
%----------------------------------------------------------------------------------------

\roottitle{References} % Top level section
Available upon request.
\iffalse
  \inlineheadsection
      {\href{mailto:Christophe.Arviset@esa.int}{Christophe ARVISET}}
      {Head of the \href{http://www.cosmos.esa.int/web/esdc}{Science Archives Unit}, \href{http://www.esa.int/About\_Us/ESACESAC}{ESAC}}

      \inlineheadsection{}{}

  \inlineheadsection
      {\href{mailto:i_koufoudakis@c-gaia.gr}{Yannis KOUFOUDAKIS}}
      {Managing Director, \href{https://www.c-gaia.gr/}{GAIA}.
        \\ GAIA is a 50 MEuro turnover company running the informational systems handling
        practically all EU-agricultural subsidies in Greece.}
\fi
\spacedhrule{1.6em}{-0.4em} % Horizontal rule - the first bracket is whitespace before and the second is after

%----------------------------------------------------------------------------------------
%	PUBLICATIONS SECTION
%----------------------------------------------------------------------------------------

\roottitle{Publications} % Top level section

  \paragraph{International Standards}
  \begin{description}
  \item[2015] Markus Demleitner, Paul Harrison, Marco Molinaro, Gretchen Greene, Theresa Dower, Menelaos Perdikeas,
    ``IVOA Registry Relational Schema'' International Virtual Observatory Alliance. Latest version available at:
    \href{http://www.ivoa.net/documents/RegTAP/}{http://www.ivoa.net/documents/RegTAP/}. Arxiv link:
    \href{https://arxiv.org/abs/1510.02275}{https://arxiv.org/abs/1510.02275}.
  \end{description}

  \paragraph{Books and reference works}
  \begin{description}
  \item[2001] Iakovos S. Venieris, Menelaos K. Perdikeas
    \href{http://onlinelibrary.wiley.com/doi/10.1002/0471219282.eot257/full}{``Distributed Intelligent Network''} article in the
    \href{http://eu.wiley.com/WileyCDA/WileyTitle/productCd-0471369721.html}{``Encyclopedia of Telecommunications''} ISBN: 0-471-36972, J. Proakis ed. John Wiley. 2002. pp. 719-729.
  \item[2000] Co-author in \href{http://eu.wiley.com/WileyCDA/WileyTitle/productCd-0471623792.html}{``Object Oriented Software Technologies in Telecommunications: from theory to practice''}, edited by I.Venieris, F.Zizza, T. Magedanz. Published by John Wiley \& Sons LTD, Chichester, UK, April 2000.
  \item[1999] M. K. Perdikeas, O. I. Pyrovolakis, F. G. Chatzipapadopoulos and I. S. Venieris, ``Service Design in Distributed Intelligent Networks'' in ``On the Way to the Information Society --- A Retrospective View on 5 Years of ACTS IS\&N Research'' Baltzer press, 1999.
  \end{description}

  
  \paragraph{Papers in International, peer-reviewed Journals}
  \begin{description}
  \item[2001] M.K. Perdikeas and I.S. Venieris,  ``Parlay-based Service Engineering in a Converged Internet-PSTN Environment'' \href{http://www.sciencedirect.com/science/journal/13891286}{Computer Networks (Elsevier)},
    vol. 35, Issue 5, April 2001, pp. 565--578
  \item[2000]F. G. Chatzipapadopoulos, M. K. Perdikeas and I. S. Venieris, ``Mobile Agent and CORBA Technologies in the Broadband Intelligent Network'', \href{http://www.comsoc.org/commag}{IEEE Communications Magazine.}, Vol. 38, Issue 6, pp. 116--124
    \item[1999] M.K. Perdikeas, F.G. Chatzipapadopoulos, I.S. Venieris and G. Marino, ``Mobile Agent Standards and Available Platforms'', \href{http://www.sciencedirect.com/science/journal/13891286}{Computer Networks (Elsevier)}, vol. 31, Issue 19, August 1999, pp. 1999--2016
  \end{description}
\paragraph{Papers in International Conferences, Workshops etc. (non-exclusive list)}
\begin{description}
\item[2019] J.-U. Ness et al. ``Towards a better coordination of Multimessenger observations: VO and future developments''
  (\href{https://arxiv.org/abs/1903.10732}{https://arxiv.org/abs/1903.10732})---Proceedings article to \nth{12} INTEGRAL conference and \nth{1} AHEAD Gamma-ray Workshop. Submitted to Journal of the Italian Astronomical Society
\item[2014] Arviset C., Perdikeas M., Osuna P. and Gonzalez J., ``The Euro-VO Registry, re-engineering the back-end'' (\href{http://adsabs.harvard.edu/abs/2015ASPC..495..457A}{http://adsabs.harvard.edu/abs/2015ASPC..495..457A})---presented in the \href{http://www.adass2014.org/announcements_en.php}{24\textsuperscript{th} annual conference on Astronomical Data Analysis Software and Systems, Calgary, Canada}, published in the
  \href{http://aspbooks.org/custom/publications/paper/495-0457.html}{ASP Conference Series, Vol. 495, Astronomical Society of the Pacific}.
  
\item[2011] Theodore Zahariadis, Menelaos Perdikeas, Fotis Chatzipapadopoulos, Javier Lucio Ruiz Andino,
  Maria Angeles Barba Rodriguez: Middleware for energy aware appliances---\nth{2} Workshop on eeBuildings Data Models, Sofia Antipolis, France
\item[2010] Menelaos Perdikeas, Theodore Zahariadis, and Pierre Plaza: The BeyWatch Conceptual Model for Demand-Side Management---E-Energy conference, October 14--15 2010, Athens, Greece
\item[2001] I. S. Venieris; T. Magedanz; M. Perdikeas; L. Hagen: Enhancing Parlay with mobile code technologies---IEEE 2001 Intelligent Network Workshop, 6-9 May 2001, Boston MA, pp. 287--299
\item[2001] M.K.Perdikeas et al. Realizing Distributed Intelligent Networks Based on Distributed Object and Mobile Agent Technologies---\nth{1} International Conference on Networking, ICN 2001, LNCS 2094, Colmar, France July 9-13, 2001, Proceedings Part II, pp. 488--496
\item[1999] M.K.Perdikeas et al. An evaluation study of mobile agent technology: standardization, implementation and evolution---IEEE International Conference on Multimedia Computing and Systems, July 1999, Florence, pp. 287--291, vol.2
%\item[1999] Designing Advanced Services for Distributed Intelligent Broadband Networks
%  Pyrovolakis O., Chatzipapadopoulos F., Perdikeas M. and Venieris Iakovos S.
%  ---Proceedings of the 7\textsuperscript{th} International Conference on Advances in Communications and Control (COMCON7)
% \item \ldots and eight (8) others in previous years including a couple of IEEE and \href{http://www.icin.co.uk/}{ICIN}
  conferences.
  
\end{description}

\spacedhrule{1.6em}{-0.4em} % Horizontal rule - the first bracket is whitespace before and the second is after

% %----------------------------------------------------------------------------------------
% %	INTERESTS SECTION
% %----------------------------------------------------------------------------------------
% 
% \roottitle{Interests} % Top level section
% 
% \inlineheadsection % Special section that has an inline header with a 'hanging' paragraph
% {}
% {Optimizing my environment, setup and workflow: Emacs, JavaScript, ReactJS, Linux, i3 window manager, \LaTeX; Math, Chess, boardgames.}

%----------------------------------------------------------------------------------------

\end{document}
